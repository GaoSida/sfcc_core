\begin{abstract}

\reminder{
    成文后需要修改!
}

这篇文章是清华大学2016年数据挖掘课程的大作业报告。
我们小组选择参加Kaggle竞赛:三藩市犯罪分类(San Francisco Crime Classification)作为本次大作业的项目。
在该项目中,我们通过学习部分标记了犯罪类型的时间、地点数据,对另一部分未标记的时间、地点数据预测犯罪类型。
我们利用已有的数据构造了一些更高层次的特征,并使用多种模型进行预测和评价。
我们发现,我们构造的特征几乎都出现了对于训练集数据过拟合的问题,导致对于测试集的预测产生了严重的负面效果。
最终,预测效果最好的方法为,使用原始特征训练Xgboost模型进行预测,该结果在排行榜上的排名为81名(共2335支队伍,约前4\%)。

\end{abstract}
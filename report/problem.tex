\section{问题描述}
\label{sec:problem}

\subsection{数据集}

本次Kaggle比赛使用的数据集来自三藩市警察局的犯罪事件数据库,包括了2003年1月1日至2015年5月13日的878049条犯罪记录(共4510天,每天平均发生约195起犯罪)。训练集和测试集使用隔周的方式划分,即从第一周开始,奇数周作为训练集,偶数周作为测试集。

在数据中,犯罪细节描述(Description)和犯罪解决方式(Resolution)两个字段只有训练集才有,因此我们无法利用这些信息。
除此之外,我们可以利用的字段如下:

\begin{itemize}
    \item \textbf{Date:}犯罪案件的时间戳,包括年、月、日、时、分、秒。
    \item \textbf{DayofWeek:}
    犯罪案件发生在星期几。
    \item \textbf{PdDistrict:}
    犯罪案件发生在哪个警局的管辖地区。
    \item \textbf{Address:}
    犯罪案件发生的街道地址。
    \item \textbf{X:}
    犯罪案件发生地点的经度。
    \item \textbf{Y:}
    犯罪案件发生地点的纬度。
\end{itemize}

\subsection{预测和评价}

在训练集中,额外给出了一个字段:

\begin{itemize}
    \item \textbf{Category:}犯罪类型。
\end{itemize}

该字段是预测的目标。数据集中,一共有39种犯罪类型。
在测试集上的预测结果并不是直接给出某一种确定的犯罪类型,而是需要给出该事件、地点39种犯罪类型的概率分布。竞赛进行评测的方法是,计算多分类的对数损失(multi-class logarithmic loss),即

\begin{equation*}
    logloss=\frac{1}{N} \sum_{i=1}^N \sum_{j=1}^M y_{ij} log(p_{ij})
\end{equation*}

其中,$N$是测试集样本总数;$M$是类型总数(本问题中$M=39$);$y_{ij}$为一个标志量,当样本$i$属于类型$j$时为1,否则为0;$p_{ij}$是模型对样本$i$属于类型$j$预测的概率。可见,当预测完全准确,即$p_{ij}=1$时,对损失的贡献为0;预测概率$p_{ij}$越小,对损失的贡献越大。因此,这是一个合理的对于多分类问题的评价标准。

在实验过程中,由于比赛每天限制5次提交,因此我们需要在训练集通过交叉验证来评价预测方法(特征和模型)。在交叉验证中,我们同样使用$logloss$来评价结果。
实验中,我们直接调用了Python的sklearn包中的实现。


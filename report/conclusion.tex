\section{结论}
\label{sec:conclusion}
通过本次参加Kaggle比赛的经历,我们学习到了很多数据挖掘的经验。在面对实际的数据挖掘问题的时候,首先要多尝试
不同的模型,来得出哪些模型比较适合这个问题。另外,不能闭门造车地纯做特征工程,因为无效的特征不仅会降低最终模型
的准确率,而且会增加模型的训练时间,因此增加特征值之后需要用模型对其进行检测,摒弃其中无效的部分;也不能纯凭自己
的想法添加特征,应当结合对数据的分析;还有需要去考虑加入的特征当前使用的模型的表达能力是否能够分析,比如说数据有
A、B、C三列,当A+B+C>100时是正例,其他情况是负例,这种特征对于决策树模型可能就比较难以分析出来。另外,在模型调
参的过程中,要去思考模型的特性,贴合实际问题去调整参数,如果训练速度较快,可以使用自动的调参工具来进行参数的选择。
最后就是要相信自己分折检验的结果,当出现提交结果与本地分折测试结果相差较大的时候,通常是自己在某些方面翻了错误,
应当将其找出并修复。